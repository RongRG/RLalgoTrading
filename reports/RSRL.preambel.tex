\documentclass[a4paper]{article}

\usepackage[latin1]{inputenc}
%\usepackage[mathcal]{euscript}
\usepackage{color}
%\usepackage{listings}
%\usepackage{ngerman}
\usepackage{amsmath,amssymb,amsfonts,bm,mathrsfs}
\usepackage{graphics}
\usepackage{graphicx}

\usepackage{algorithm}
%\usepackage{algorithmic}
%\usepackage{algorithmicx}
\usepackage[noend]{algpseudocode}

%\usepackage{mathrsfs}
\usepackage{array}
%\usepackage{bm}
%\usepackage{bbm}
\usepackage{verbatim}
\usepackage{hyperref}
\usepackage{wrapfig}


% Todo notes
%\usepackage[colorlinks]{hyperref}
\usepackage{soul}
\usepackage[colorinlistoftodos]{todonotes}
\makeatletter
 \if@todonotes@disabled
 \newcommand{\hlfix}[2]{#1}
 \else
 \newcommand{\hlfix}[2]{\texthl{#1}\todo{#2}}
 \fi
 \makeatother


\usepackage[authoryear,round]{natbib}

\usepackage{longtable}

% Special things
\def\TReg{\textsuperscript{\textregistered}}
\def\TCop{\textsuperscript{\textcopyright}}
\def\TTra{\textsuperscript{\texttrademark}}
\def\Deg{$^\circ\,$}

% Text tricks
\newcommand{\qu}[1]{``#1''}

% Common functions, distributions and policies
\newcommand{\indiroot}[0]{\vartheta}
\newcommand{\indidist}[0]{\zeta}
\newcommand{\indipol}[0]{\tau}
\newcommand{\indifun}[0]{\phi}
\newcommand{\facfun}[0]{\varphi}
\newcommand{\basefun}[0]{\psi}
\newcommand{\indi}[0]{\indiroot}
\newcommand{\KLdiv}[0]{D_{\text{KL}}}
\newcommand{\rnd}[2]{{\textstyle\frac{d#1}{d#2}}}

% Sets
\def\all{:}
\def\R{\mathrm{I\kern-0.4ex R}}
\def\N{\mathrm{I\kern-0.4ex N}}
\newcommand{\Set}[1]{\mathcal{#1}}
\newcommand{\Bx}[0]{\Set B(\Set X)}
\newcommand{\Ba}[0]{\Set B(\Set A)}
\newcommand{\Bxa}[0]{\Set B(\Set X \times \Set A)}
\newcommand{\Lt}[1]{L^2(#1)}
\newcommand{\Lx}[0]{\Lt{\Set X,\zeta}}
\newcommand{\Lxa}[0]{\Lt{\Set X \times \Set A, \indi}}
\newcommand{\Lz}[0]{\Lt{\Set Z,\indi}}

% Math declarations
\newcommand{\sgn}{\mathop{\mathrm{sgn}}}
\newcommand{\mat}[1]{\mathbf{#1}}
\newcommand{\ve}[1]{\bm #1} %{\bm{#1}}
\newcommand{\grad}[0]{\nabla}
\newcommand{\dom}[0]{\mathbf{dom}\,}
\newcommand{\trace}[0]{\mathbf{tr}}
\newcommand{\shallbe}[0]{\stackrel{!}{=}}
\newcommand{\statetrans}[1]{\stackrel{#1}{\to}}
\newcommand{\frobenius}[0]{\text{F}}
\newcommand{\normdist}[0]{\mathcal{N}}

\newcommand{\wsp}[3]{\langle #1 , #2 \rangle_{ #3 }}
\newcommand{\isp}[2]{\wsp{#1}{#2}{\indidist\indipol}}
\newcommand{\kernel}[0]{\kappa}
\newcommand{\banach}[1]{\mathcal{B}(\Set #1)}
\newcommand{\eqgap}[0]{\vspace{.1cm}}
\newcommand{\op}[1]{\hat #1}
\newcommand{\projectop}[2]{\op \Pi^{#2}_{#1}}
\newcommand{\prop}[0]{\projectop{\indi}{\basefun}}
\newcommand{\propphi}[0]{\projectop{\zeta}{\phi}}
\newcommand{\proppsi}[0]{\projectop{\indiroot}{\psi}}
\newcommand{\meanpol}[1]{\op \Gamma{\kern-0.5ex}_{#1}}
\newcommand{\meanaction}[0]{\meanpol{\tau}}
\newcommand{\meanpolicy}[0]{\meanpol{\pi}}
%\newcommand{\meanaction}[0]{\op \Gamma{\kern-0.5ex_\tau}}
%\newcommand{\meanpolicy}[0]{\op \Gamma{\kern-0.5ex_\pi}}
\newcommand{\subgrad}[1]{\grad\kern-0.5ex_{#1}}


% Operators
%\def\E{\mathrm{I\kern-0.4ex E}}
\def\E{\mathbb{E}}
\def\V{\mathbb{V}}
\def\C{\mathbb{C}}
\def\notexists{\mathrm{| \kern-1.0ex \exists}}
\DeclareMathOperator*{\argmax}{arg\,max}
\DeclareMathOperator*{\argmin}{arg\,min}
%\DeclareMathOperator*{\exp}{exp}
\DeclareMathOperator*{\setunit}{\cup}
%\DeclareMathOperator*{\smallsum}{ {\textstyle\sum\limits} }

\newcommand{\smallprod}[2]{ {\textstyle \prod\limits_{\scriptscriptstyle #1}^{\scriptscriptstyle #2}} }
\newcommand{\smallsum}[2]{ {\textstyle \sum\limits_{\scriptscriptstyle #1}^{\scriptscriptstyle #2}} }
\newcommand{\smallhalf}[0]{{\textstyle\frac{1}{2}}}
\newcommand{\smallfrac}[2]{ {\textstyle \frac{#1}{#2}} }
\newcommand{\stack}[1]{\!\!\begin{array}{c}\scriptstyle #1\end{array}\!\!}
\newcommand{\brs}[1][-1mm]{\\[#1]\scriptstyle}

% Formats
\newcommand{\algcomment}[1]{{\scriptsize\sc // #1}}
\newcommand{\gap}[0]{\vspace{0.25cm}}
\newcommand{\antigap}[0]{\vspace{-0.25cm}}
\newcommand{\closetoequation}[0]{\begin{eqnarray*}
\frac{\partial \pi_a(\ve q)}{\partial \beta} 
	&=& q_a \pi_a(\ve q)
		- \pi_a(\ve q) \; \ve q^\top \ve \pi(\ve q) \\
\frac{\partial}{\partial \beta} \sum_{t=1}^n \ve q_t^\top \ve \pi(\ve q_t)
	&=& \sum_{t=1}^n \Big( \ve q_t \otimes \ve q_t \Big)^\top \ve \pi(\ve q_t)
		- \sum_{t=1}^n \Big( \ve q_t^\top \ve \pi(\ve q_t) \Big)^2
\end{eqnarray*}

  \vspace{-0.6cm} 

  \hspace{-0.6cm}
}

\newenvironment{proof}
	{\noindent\textbf{Proof:}}
	{\vspace{-.5cm} \begin{flushright} $\Box$ \end{flushright} }


% Theorem corner
\newtheorem{theorem}{Theorem}
\newtheorem{assumption}[theorem]{Assumption}
\newtheorem{definition}[theorem]{Definition}
\newtheorem{proposition}[theorem]{Proposition}
\newtheorem{lemma}[theorem]{Lemma}
\newtheorem{corollary}[theorem]{Corollary}
\newtheorem{remark}[theorem]{Remark}

% Array definition
\renewcommand{\arraystretch}{2.125}


% Paper format
\evensidemargin 0.8cm
\oddsidemargin  0.8cm
\topmargin -2cm
\setlength\textwidth{14.2cm} 
\setlength\textheight{25cm}

% Code style
\usepackage[procnames]{listings}
\definecolor{mykeywords}{RGB}{0,0,255}
\definecolor{comments}{RGB}{96,96,96}
\definecolor{red}{RGB}{160,0,0}
\definecolor{green}{RGB}{0,150,0}
\definecolor{light-gray}{gray}{0.95}
\lstset{language=Python, 
        basicstyle=\ttfamily\small, 
        keywordstyle=\color{mykeywords},
        commentstyle=\color{comments},
        stringstyle=\color{red},
        showstringspaces=false,
        identifierstyle=\color{black},
        procnamekeys={def,class},
        tabsize=4,
        backgroundcolor=\color{light-gray}
}

\def\packenumtopsep{0.75\topsep}
\newenvironment{packenum}{%
\vspace{-\packenumtopsep}%
\begin{enumerate}%
\setlength\itemsep{-1.4mm}%
\setlength\itemindent{-3mm}%
}{%
\end{enumerate}%
\vspace{-\packenumtopsep}%
}
\newcommand{\smallparagraph}{\vspace{2mm}\noindent}
\newcommand\new[1]{{\color{blue}#1}}

\usepackage{hyperref}