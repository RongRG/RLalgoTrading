\documentclass[a4paper]{article}

\usepackage[latin1]{inputenc}
%\usepackage[mathcal]{euscript}
\usepackage{color}
%\usepackage{listings}
%\usepackage{ngerman}
\usepackage{amsmath,amssymb,amsfonts,bm,mathrsfs}
\usepackage{graphics}
\usepackage{graphicx}

\usepackage{algorithm}
%\usepackage{algorithmic}
%\usepackage{algorithmicx}
\usepackage[noend]{algpseudocode}

%\usepackage{mathrsfs}
\usepackage{array}
%\usepackage{bm}
%\usepackage{bbm}
\usepackage{verbatim}
\usepackage{hyperref}
\usepackage{wrapfig}


% Todo notes
%\usepackage[colorlinks]{hyperref}
\usepackage{soul}
\usepackage[colorinlistoftodos]{todonotes}
\makeatletter
 \if@todonotes@disabled
 \newcommand{\hlfix}[2]{#1}
 \else
 \newcommand{\hlfix}[2]{\texthl{#1}\todo{#2}}
 \fi
 \makeatother


\usepackage[authoryear,round]{natbib}

\usepackage{longtable}

% Special things
\def\TReg{\textsuperscript{\textregistered}}
\def\TCop{\textsuperscript{\textcopyright}}
\def\TTra{\textsuperscript{\texttrademark}}
\def\Deg{$^\circ\,$}

% Text tricks
\newcommand{\qu}[1]{``#1''}

% Common functions, distributions and policies
\newcommand{\indiroot}[0]{\vartheta}
\newcommand{\indidist}[0]{\zeta}
\newcommand{\indipol}[0]{\tau}
\newcommand{\indifun}[0]{\phi}
\newcommand{\facfun}[0]{\varphi}
\newcommand{\basefun}[0]{\psi}
\newcommand{\indi}[0]{\indiroot}
\newcommand{\KLdiv}[0]{D_{\text{KL}}}
\newcommand{\rnd}[2]{{\textstyle\frac{d#1}{d#2}}}

% Sets
\def\all{:}
\def\R{\mathrm{I\kern-0.4ex R}}
\def\N{\mathrm{I\kern-0.4ex N}}
\newcommand{\Set}[1]{\mathcal{#1}}
\newcommand{\Bx}[0]{\Set B(\Set X)}
\newcommand{\Ba}[0]{\Set B(\Set A)}
\newcommand{\Bxa}[0]{\Set B(\Set X \times \Set A)}
\newcommand{\Lt}[1]{L^2(#1)}
\newcommand{\Lx}[0]{\Lt{\Set X,\zeta}}
\newcommand{\Lxa}[0]{\Lt{\Set X \times \Set A, \indi}}
\newcommand{\Lz}[0]{\Lt{\Set Z,\indi}}

% Math declarations
\newcommand{\sgn}{\mathop{\mathrm{sgn}}}
\newcommand{\mat}[1]{\mathbf{#1}}
\newcommand{\ve}[1]{\bm #1} %{\bm{#1}}
\newcommand{\grad}[0]{\nabla}
\newcommand{\dom}[0]{\mathbf{dom}\,}
\newcommand{\trace}[0]{\mathbf{tr}}
\newcommand{\shallbe}[0]{\stackrel{!}{=}}
\newcommand{\statetrans}[1]{\stackrel{#1}{\to}}
\newcommand{\frobenius}[0]{\text{F}}
\newcommand{\normdist}[0]{\mathcal{N}}

\newcommand{\wsp}[3]{\langle #1 , #2 \rangle_{ #3 }}
\newcommand{\isp}[2]{\wsp{#1}{#2}{\indidist\indipol}}
\newcommand{\kernel}[0]{\kappa}
\newcommand{\banach}[1]{\mathcal{B}(\Set #1)}
\newcommand{\eqgap}[0]{\vspace{.1cm}}
\newcommand{\op}[1]{\hat #1}
\newcommand{\projectop}[2]{\op \Pi^{#2}_{#1}}
\newcommand{\prop}[0]{\projectop{\indi}{\basefun}}
\newcommand{\propphi}[0]{\projectop{\zeta}{\phi}}
\newcommand{\proppsi}[0]{\projectop{\indiroot}{\psi}}
\newcommand{\meanpol}[1]{\op \Gamma{\kern-0.5ex}_{#1}}
\newcommand{\meanaction}[0]{\meanpol{\tau}}
\newcommand{\meanpolicy}[0]{\meanpol{\pi}}
%\newcommand{\meanaction}[0]{\op \Gamma{\kern-0.5ex_\tau}}
%\newcommand{\meanpolicy}[0]{\op \Gamma{\kern-0.5ex_\pi}}
\newcommand{\subgrad}[1]{\grad\kern-0.5ex_{#1}}


% Operators
%\def\E{\mathrm{I\kern-0.4ex E}}
\def\E{\mathbb{E}}
\def\V{\mathbb{V}}
\def\C{\mathbb{C}}
\def\notexists{\mathrm{| \kern-1.0ex \exists}}
\DeclareMathOperator*{\argmax}{arg\,max}
\DeclareMathOperator*{\argmin}{arg\,min}
%\DeclareMathOperator*{\exp}{exp}
\DeclareMathOperator*{\setunit}{\cup}
%\DeclareMathOperator*{\smallsum}{ {\textstyle\sum\limits} }

\newcommand{\smallprod}[2]{ {\textstyle \prod\limits_{\scriptscriptstyle #1}^{\scriptscriptstyle #2}} }
\newcommand{\smallsum}[2]{ {\textstyle \sum\limits_{\scriptscriptstyle #1}^{\scriptscriptstyle #2}} }
\newcommand{\smallhalf}[0]{{\textstyle\frac{1}{2}}}
\newcommand{\smallfrac}[2]{ {\textstyle \frac{#1}{#2}} }
\newcommand{\stack}[1]{\!\!\begin{array}{c}\scriptstyle #1\end{array}\!\!}
\newcommand{\brs}[1][-1mm]{\\[#1]\scriptstyle}

% Formats
\newcommand{\algcomment}[1]{{\scriptsize\sc // #1}}
\newcommand{\gap}[0]{\vspace{0.25cm}}
\newcommand{\antigap}[0]{\vspace{-0.25cm}}
\newcommand{\closetoequation}[0]{\begin{eqnarray*}
\frac{\partial \pi_a(\ve q)}{\partial \beta} 
	&=& q_a \pi_a(\ve q)
		- \pi_a(\ve q) \; \ve q^\top \ve \pi(\ve q) \\
\frac{\partial}{\partial \beta} \sum_{t=1}^n \ve q_t^\top \ve \pi(\ve q_t)
	&=& \sum_{t=1}^n \Big( \ve q_t \otimes \ve q_t \Big)^\top \ve \pi(\ve q_t)
		- \sum_{t=1}^n \Big( \ve q_t^\top \ve \pi(\ve q_t) \Big)^2
\end{eqnarray*}

  \vspace{-0.6cm} 

  \hspace{-0.6cm}
}

\newenvironment{proof}
	{\noindent\textbf{Proof:}}
	{\vspace{-.5cm} \begin{flushright} $\Box$ \end{flushright} }


% Theorem corner
\newtheorem{theorem}{Theorem}
\newtheorem{assumption}[theorem]{Assumption}
\newtheorem{definition}[theorem]{Definition}
\newtheorem{proposition}[theorem]{Proposition}
\newtheorem{lemma}[theorem]{Lemma}
\newtheorem{corollary}[theorem]{Corollary}
\newtheorem{remark}[theorem]{Remark}

% Array definition
\renewcommand{\arraystretch}{2.125}


% Paper format
\evensidemargin 0.8cm
\oddsidemargin  0.8cm
\topmargin -2cm
\setlength\textwidth{14.2cm} 
\setlength\textheight{25cm}

% Code style
\usepackage[procnames]{listings}
\definecolor{mykeywords}{RGB}{0,0,255}
\definecolor{comments}{RGB}{96,96,96}
\definecolor{red}{RGB}{160,0,0}
\definecolor{green}{RGB}{0,150,0}
\definecolor{light-gray}{gray}{0.95}
\lstset{language=Python, 
        basicstyle=\ttfamily\small, 
        keywordstyle=\color{mykeywords},
        commentstyle=\color{comments},
        stringstyle=\color{red},
        showstringspaces=false,
        identifierstyle=\color{black},
        procnamekeys={def,class},
        tabsize=4,
        backgroundcolor=\color{light-gray}
}

\def\packenumtopsep{0.75\topsep}
\newenvironment{packenum}{%
\vspace{-\packenumtopsep}%
\begin{enumerate}%
\setlength\itemsep{-1.4mm}%
\setlength\itemindent{-3mm}%
}{%
\end{enumerate}%
\vspace{-\packenumtopsep}%
}
\newcommand{\smallparagraph}{\vspace{2mm}\noindent}
\newcommand\new[1]{{\color{blue}#1}}

\usepackage{hyperref}

\title{Risk-sensitive Distributional Reinforcement Learning for Algorithmic Trading}
\author{Rong Guo, Egemen Okur, Arsham Afsardeir, Klaus Obermayer}

% ==============================================================================

\begin{document}
\maketitle

\listoftodos

% ==============================================================================
\section{Domain Background}

\begin{center}
\begin{longtable}{| m{4cm} | m{5cm} | m{5cm} | }
\hline
\textbf{Paper} & \textbf{Summary} & \textbf{Drawbacks w.r.t ours}\\
\hline
\hline
RSQ \citep{shen_risk-sensitive_2014, shen_risk-averse_2014} & Utility function over TD error, convergence and optimality guarantees & Tabular Q; Trading on limit order market \\
\hline
Ensemble actor-critic \citep{yang_deep_2020} & Ensemble trained by returns but the winner is selected by Sharp ration & Add-hoc risk strategies\\
\hline
Exponential Bellman Equation \citep{fei_exponential_nodate} & Theoretical guarantees; An instantiation of distributional RL through the MGF of rewards & theoretical paper, no application \\
\hline
Risk-senstive Distributional Q \citep{bodnar_quantile_2020, dabney_implicit_2018} & Various risk measures; Discrete and continuous action space Q-learning & robotics\\
\hline
\end{longtable}
\end{center}


% ==============================================================================
\section{Problem and Solution Statement}

\subsection{Portfolio Optimization as A Reinforcement Learning Problem}

In many financial decision making tasks, expected returns fail to describe the real outcomes. Outcomes must be
reasoned through considering not only the expected value, but also volatility and risk. The outcome (return) is a random variable.\\

Deep reinforcement learning has shown to be a useful too for portfolio optimization \citep{zhang_deep_2020}. The recent distributional reinforcement learning \citep{bellemare_distributional_2017, dabney_distributional_2018} endows policies with risk-sensitive strategies, by taking the entire value distributions into account.\\

\textit{Technical novelty? not all original, but are rather an interesting use-case/application of existing methods and are a useful contribution.}
% ==============================================================================
\section{Dataset \& Input \& Metric}
\begin{enumerate}
\item
	\todo[inline]{Environment/Experiments.}
	\begin{itemize}
	\item
	portfolio allocation problem
	\end{itemize}

\item
	\todo[inline]{Benchmark model \& results, which contextualises existing methods/knowledge in the domain.}
	\begin{itemize}
	\item
	SOTA: How are the existing work related to our methods? 
	\item
	Experiments contrasting risk-neutral with risk-sensitive RL methods
	\end{itemize}
\item
	\todo[inline]{Metric: objectively compare the solution model with the benchmark model.}
	\begin{itemize}
	\item
	Returns vs. Risk
	\item
	Sharp ration, CVaR, etc.
	\item
	Theoretical guarantees and insights on convergence and optimatliy
	\end{itemize}
\end{enumerate}

% ==============================================================================
\section{Theoretical Workflow}

\subsection{End-to-end formalization}

\subsection{Using Distributional RL for Risk Awareness}

\subsection{Convergence Analysis}

% ==============================================================================
\section{Network Architecture}
\begin{itemize}
\item
MLP
\end{itemize}


%\newpage
% ==============================================================================
%\def\FormatName#1{\IfSubStr{#1}{B\"ohmer}{\textbf{#1}}{#1}}
\def\FormatName#1{#1}
\bibliographystyle{highlight}
{\footnotesize\bibliography{references}}

\end{document}
